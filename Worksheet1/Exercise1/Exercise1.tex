The unix commandline provides a number of preset commands. Examples of some are given below:
(a) mkdir compphys
This command creates a directory, within the current directory, called compphys.

(b) cd compphys
This command changes the current directory to compphys, provided that compphys is
a directory within the current directory.

(c) cat > file1.txt [rtn] this is my first file [rtn][ctrl-c]
This command creates a file called file1.txt. Upon pressing [rtn], a user can input
their desired content for file1.txt. Pressing [rtn] again will add this content to
the file. The key combination [ctrl-c] will shut the file and return the user to
the commandline.

(d) ls
This command alphabetically lists the names of the files in the current directory.

(e) more file1.txt
This command prints the content of file1.txt to the console.

(f) xclock &
This command starts an xclock process. The & in the command means the process is performed
in the background, allowing the user to keep using the same console window. This process
creates a clock in the current system's GUI.

(g) whoami
This command prints the username of the current user.

(h) man ls
This command opens the manual page for the ls command.

(i) top
This command gives a list of the processes currently being performed by all users.

(j) ps -u [username]
This command provides a list of all processes being performed by [username].

(k) kill [PID]
This command kills a process with the specific ProcessID.
For example, if an instance of xclock has the PID 65934, the command kill 65934
terminates the xclock process.