The Silver ratio is a quantity given by (SILVER RATIO) where (0) is the Silver Ratio.
This quantity is also known as the Golden Ratio Conjugate as (golden - silver) where (1)
is the golden ratio.
The following is true for the golden ratio:(stuff)
so it can be shown that (stuff)
This defines a recursion relation for the silver ratio. This relation can be used as a
recursive function in C++ to calculate successive powers of the silver ratio.
The results are shown (below/inthefile?)
When the recursive formula uses float variables in its calculation, it can be seen to lose
accuracy against the direct multiplication within 5 powers of phi.